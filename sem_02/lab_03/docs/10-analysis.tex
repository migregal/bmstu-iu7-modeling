\chapter{Теоретическая часть}

\section{Методы генерации псевдослучайных последовательностей}

\subsection{Линейный конгруэнтный метод}

Данный метод использует 4 числа:

\begin{itemize}
    \item $m > 0$ - модуль;
    \item $0 \le a \le m$ - множитель;
    \item $0 \le c \le m$ - приращение;
    \item $0 \le X_0 \le m$ - начальное число;
\end{itemize}

Последовательность случайных чисел генерируется следующим образом:

\begin{equation}
    X_{n+1} = (a X_n + c) \mod m
\end{equation}

\subsection{Табличный метод}

Табличные методы опираются на заранее составленные таблицы, содержащие некореллированные числа, т.е. числа не зависящие друг от друга.

От содержимого таблицы зависит качество генерируемой последовательности.

\section{Критерий оценки}

В качестве критерия оценки были выбран критерий монотонности, опирающийся на критерий \(\chi^2\).

\section{Критерий \(\chi^2\)}

Это один из самых известных статистических критериев,  также это основной метод, используемый в сочетании с дргуими критериями.
С помощью этого критерия можно узнать, удовлетворяет ли генератор случайных чисел требованию равномерного распределения или нет.
Для оценки по этому критерию необходимо вычислить статистику V по формуле:

\begin{equation}
    V = \frac{1}{n}\sum_{s=1}^k(\frac{Y_s^2}{p_s}) - n
\end{equation}

где $n$ --- количество независимых испытаний, $k$ --- количество категорий, $Y_s$ --- число наблюдений, которые действительно относятся к категории $S$,  $p_s$ --- вероятность того, что каждое наблюдение относится к категории $S$.

\section{Критерий монотонности}

Критерий применяется для проверки рас- пределения длин монотонных подпоследовательностей в последовательностях вещественных чисел.

Рассмотрим пример. Допустим, у нас есть следующая выборка: 

$$0.8, 0.13, 0.5, 0.27, 0.43, 0.85, 0.63, 0.74$$

В данной последовательности найдено 4 отрезка возрастания: 

$$[0.8], [0.5], [0.43, 0.85], [0.74]$$

Таким образом, мы имеем три отрезка длиной 1 и один отрезок длиной 2.

Смежные отрезки не являются независимыми, поэтому необходимо <<выбросить>> элемент, который следует непосредственно за серией. Таким образом, когда $U_j$ больше $U_{j+1}$, начнем следующую серию с $U_{j+2}$. 

В таком случае, после подсчета количества отрезков возрастания с различной длиной, можем использовать критерий \(\chi^2\) со следующими вероятностями:

\begin{equation}
    \begin{cases}
        p_r = \frac{1}{r!} - \frac{1}{(r + 1)!}, r < t \\
        p_t = \frac{1}{r!}, r \ge t
    \end{cases}
\end{equation}