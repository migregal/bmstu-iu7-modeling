\chapter{Теоретическая часть}

\section{Равномерное распределение}

Говорят, что случайная величина \(X\) имеет равномерное распределение на отрезке \([a; b]\), если её функция плотности имеет вид

\begin{equation}
    f(x) = \begin{cases}
        \frac{1}{b - a}, a \le x \le b \\
        0, \text{иначе}
    \end{cases}
\end{equation}

Обозначается \(X \sim R[a, b] \).

Соответствующая функция распределения:

\begin{equation}
    F(x) = \begin{cases}
        0, a < x \\
        \frac{x - a}{b - a}, a \le x \le b \\
        1, x > b
    \end{cases}
\end{equation}

\section{Распределение Пуассона}

Говорят, что случайная величина \(X\) распределена по закону Пуассона с параметром \(\lambda > 0\), если она принимает значения \(0, 1, 2, \dots\) с вероятностями

\begin{equation}
    P\{X = k\} = \frac{\lambda^k}{k!}*e^{-\lambda}, k \in {0, 1, 2, \dots}
\end{equation}

Обозначается \(X \sim \Pi(\lambda)\).

Функция плотности распределения имеет вид:
\begin{equation}
    P\{x = k\} = \frac{\lambda^k}{k!}*e^{-\lambda}, k \in {0, 1, 2, \dots}
\end{equation}

Тогда соответствующая функция распределения имеет вид:
\begin{equation}
    F(x) = P\{X < x\}, X \sim \Pi(\lambda)
\end{equation}

\section{Формализация задачи}

\subsection{$\Delta$t модель}

Данная модельн заключается в последовательном анализе состояний всех блоков системы в момент времени $t + \Delta t$. Новое состояние определяется в соответствии с их алгоритмическим описанием с учетом действия случайных факторов. В результате этого анализа принимается решение о том, какие системные события должны имитироваться на данный момент времени. 

Основной недостаток модели: значительные затраты и вероятность пропуска события при больших $\Delta t$.

\subsection{Событийная модель}

В данной модели состояния отдельных устройств изменяются в дискретные моменты времени. При использовании событийного принципа, состояния всех блоков системы анализируются лишь в момент возникновения какого либо события. Момент наступления следующего события, определяется минимальным значением из списка событий.