\chapter{Теоретическая часть}

\section{Марковский процесс}

Случайный процесс, протекающий в системе $S$, называется Марковским, если он обладает следующим свойством: для каждого момента времени $t_0$ вероятность любого состояния системы в будущем (при $t > t_0$) зависит только от ее состояния в настоящем (при $t = t_0$) и не зависит от того, когда и каким образом система пришла в это состояние. 

Вероятность того, что в момент $t$ система будет находиться в состоянии $S_i$, называется 
вероятностью $i$-го состояния. Для любого момента $t$ сумма вероятностей всех состояний равна единице.

\section{Вычисление вероятностной константы}

Для решения поставленной задачи, необходимо составить систему уравнений Колмогорова по следующим принципам: 
\begin{itemize}
    \item в левой части каждого из уравнений стоит производная вероятности $i$-го состояния;
    \item в правой части --- сумма произведений вероятностей всех состояний, из которых идут стрелки в данное состояние, умноженная на интенсивности соответствующих потоков событий, минус суммарная интенсивность всех потоков, выводящих систему из данного состояния, умноженная на вероятность данного состояния.
\end{itemize}

\section{Пример}

Пусть дана матрица интенсивностей для системы $S$, имеющей три состояния:

\begin{equation}
  \left( {\begin{array}{ccc}
    0 & \lambda_{01} & \lambda_{02} \\
    \lambda_{10} & 0 &\lambda_{12} \\
    \lambda_{20} & \lambda_{21} & 0\\
  \end{array} } \right)
\end{equation}

\clearpage

Тогда соответствующая система уравнений Колмогорова имеет вид:

\begin{equation} \label{eq:1}
    \begin{cases}
        p_0' = - (\lambda_{01} + \lambda_{02})p_0 + \lambda_{10}p_1+\lambda_{20}p_2\\
        p_1' = - (\lambda_{10} + \lambda_{12})p_1 + \lambda_{01}p_0+\lambda_{21}p_2\\
        p_2' = - (\lambda_{20} + \lambda_{21})p_2 + \lambda_{02}p_0+\lambda_{12}p_1\\
    \end{cases}
\end{equation}

Для нахождения вероятностных констант, то есть вероятностей в стационарном режиме работы при $t -> \inf$, необходимо приравнять левые части уравнений к нулю. 

Таким образом получается система линейных уравнений. Для решения полученной системы необходимо добавить условие нормировки: 

\begin{equation}
    \sum_{i}p_i = 1
\end{equation}
 
Кроме того, необходимо так же найти время, за которое достигается вероятностная константа. 

В общем случае для решения данной задачи необходимо решить систему ОДУ \ref{eq:1} в общем виде. 

Для решения данной задачи можно воспользоваться численным методом: найдем все значения вероятности как функции времени, с шагом в некотором интервале $[t_0, t_N]$. Когда вычисленная вероятность будет равна найденной ранее вероятностной константе с точностью до заданной погрешности, можно считать что искомое время найденно.
