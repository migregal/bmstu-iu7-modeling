\chapter{Результат работы}

\section{Общий случай}

В таблице \ref{tbl:system8} приведена матрица интенсивностей для рассматриваемой системы.

\begin{table}[ht]
	\begin{center}
		\caption{Описание системы}
		\label{tbl:system8}
		\begin{tabular}{|c|c|c|c|c|c|c|c|c|}
			\hline
			Из / B & $S_1$ & $S_2$ & $S_3$ & $S_4$ & $S_5$ & $S_6$ & $S_7$ & $S_8$ \\\hline
			$S_1$ & 0.83 & 0.15 & 0.37 & 0.89 & 0.05 & 0.11 & 0.27 & 0.01 \\\hline
			$S_2$ & 0.39 & 0.86 & 0.98 & 0.84 & 0.91 & 0.93 & 0.94 & 0.06 \\\hline
			$S_3$ & 0.77 & 0.26 & 0.11 & 0.13 & 0.48 & 0.68 & 0.75 & 0.17 \\\hline
			$S_4$ & 0.12 & 0.09 & 0.70 & 0.95 & 0.96 & 0.53 & 0.76 & 0.23 \\\hline
			$S_5$ & 0.79 & 0.77 & 0.86 & 0.59 & 0.80 & 0.55 & 0.71 & 0.60 \\\hline
			$S_6$ & 0.64 & 0.32 & 0.34 & 0.04 & 0.96 & 0.86 & 0.02 & 0.21 \\\hline
			$S_7$ & 0.24 & 0.60 & 0.65 & 0.26 & 0.20 & 0.35 & 0.93 & 0.39 \\\hline
			$S_8$ & 0.61 & 0.63 & 0.27 & 0.94 & 0.81 & 0.49 & 0.12 & 0.55 \\\hline
		\end{tabular}
	\end{center}
\end{table}

На рисурке \ref{img:s8} представлен граф, описывающий связи внутри рассматриваемой системы.

\imgw{s8}{ht!}{0.8\textwidth}{Граф, описывающий систему}

На рисунке \ref{img:r8} представлен график зависимости вероятности каждого из состояний от времени.

\imgw{r8}{ht!}{0.9\textwidth}{Зависимость вероятности от времени}

В таблице \ref{tbl:res8} представлены полные результаты работы программы для рассматрвиаемой системы.

\begin{table}[ht]
	\small
	\begin{center}
		\caption{Результаты работы}
		\label{tbl:res8}
		\begin{tabular}{|c|c|c|c|c|c|c|c|c|}
			\hline
			& $S_1$ & $S_2$ & $S_3$ & $S_4$ & $S_5$ & $S_6$ & $S_7$ & $S_8$ \\\hline
			Стабильное состояние &0.150 &  0.100 &  0.129 &  0.140 &  0.145 &  0.134 &  0.141 &  0.062 \\\hline
			Время & 14.381 & 12.971 & 12.655 & 12.813 & 13.447 & 13.291 & 14.761 & 12.157 \\\hline
		\end{tabular}
	\end{center}
\end{table}

\clearpage

\section{Кольцо}

В таблице \ref{tbl:systemloop} приведена матрица интенсивностей для рассматриваемой системы.

\begin{table}[ht]
	\begin{center}
		\caption{Описание системы}
		\label{tbl:systemloop}
		\begin{tabular}{|c|c|c|c|c|}
			\hline
			Из / B & $S_1$ & $S_2$ & $S_3$ & $S_4$ \\\hline
			$S_1$ & 0.00 & 0.80 & 0.00 & 0.00 \\\hline
			$S_2$ & 0.00 & 0.00 & 0.30 & 0.00 \\\hline
			$S_3$ & 0.00 & 0.00 & 0.00 & 0.26 \\\hline
			$S_4$ & 0.53 & 0.00 & 0.00 & 0.00 \\\hline
		\end{tabular}
	\end{center}
\end{table}

На рисурке \ref{img:sloop} представлен граф, описывающий связи внутри рассматриваемой системы.

\imgw{sloop}{ht!}{0.8\textwidth}{Граф, описывающий систему}

На рисунке \ref{img:rloop} представлен график зависимости вероятности каждого из состояний от времени.

\imgw{rloop}{ht!}{0.9\textwidth}{Зависимость вероятности от времени}

В таблице \ref{tbl:resloop} представлены полные результаты работы программы для рассматрвиаемой системы.

\begin{table}[ht]
	\small
	\begin{center}
		\caption{Результаты работы}
		\label{tbl:resloop}
		\begin{tabular}{|c|c|c|c|c|}
			\hline
			& $S_1$ & $S_2$ & $S_3$ & $S_4$ \\\hline
			Стабильное состояние &  0.250 & 0.250 & 0.250 & 0.250 \\\hline
			Время & 12.171	& 13.003 & 12.171 & 13.003 \\\hline
		\end{tabular}
	\end{center}
\end{table}
