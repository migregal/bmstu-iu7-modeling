\chapter{Задание}

В информационный центр приходят клиенты через интервал времени $10 \pm 2$ минуты. Если все три имеющихся оператора заняты, клиенту отказывают в
обслуживании. Операторы имеют разную производительность и могут обеспечивать обслуживание среднего запроса пользователя за $20 \pm 5$, $40 \pm 10$ и
$40 \pm 20$, соответственно. 

Клиенты стремятся занять свободного оператора с максимальной производительностью. Полученные запросы сдаются в накопитель, откуда выбираются на обработку. На первый компьютер запросы от $1$ и $2$-ого
операторов, на второй -- запросы от 3-его. Время обработки запросов первым и 2-м компьютером равны соответственно $15$ и $30$ мин. Промоделировать процесс
обработки $300$ запросов.

Для выполнения поставленного задания необходимо создать
концептуальную модель в терминах СМО, определить эндогенные и экзогенные переменные и уравнения модели. 

За единицу системного времени выбрать $0.01$ минуты.