\chapter{Теоретическая часть}

\section{Общий вид системы}

На рисунке~\ref{img:common} представлена концептуальная модель моделируемой     системы в общем виде.

\imgw{common}{ht!}{0.9\textwidth}{Концептуальная модель в общем виде}

\section{Концептуальная модель системы в терминах СМО}

На рисунке~\ref{img:smo} представлена концептуальная модель моделируемой системы в терминах СМО.

\imgw{smo}{ht!}{0.9\textwidth}{Концептуальная модель в терминах СМО}

В процессе взаимодействия клиентов с информационным центром
возможны:
\begin{itemize}
    \item режим нормального обслуживания, т.е. клиент выбирает одного из
свободных операторов, отдавая предпочтение тому у которого меньше номер;
    \item режим отказа в обслуживании клиента, когда все операторы заняты.
\end{itemize}

\section{Переменные и уравнения имитационной модели}

Эндогенные переменные: 
\begin{itemize}
    \item время обработки задания $i$-ым оператором;
    \item время решения этого задания $j$-ым компьютером.
\end{itemize}

Экзогенные переменные: 
\begin{itemize}
    \item число обслуженных клиентов;
    \item число клиентов, получивших отказ.
\end{itemize}

Вероятность отказа в обслуживании клиента: 

$$\frac{C_{\text{отк}}}{C_{\text{отк}} + C_{\text{обсл}}}$$

где $C_{\text{отк}}$ -- количество заявок которым отказали в обслуживании, а $C_{\text{обсл}}$ -- количество обслуженных заявок.

