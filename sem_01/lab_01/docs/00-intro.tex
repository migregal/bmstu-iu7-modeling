\chapter*{Введение}
\addcontentsline{toc}{chapter}{Введение}

Моделирование –- специфический метод (методология) познания, основанный на замене
реального объекта (системы, процесса, ситуации, явления) некоторой моделью и исследовании
в дальнейшем созданной модели.

Целью данной лабораторной работы является получение навыков решения задачи Коши при помощи методов Пикара, Эйлера и Рунге--Кутта 2-го порядка.

Для достижения данной цели необходимо решить следующие задачи:

\begin{itemize}
    \item описать постановку задачи Коши;
    \item привести конкретный пример задачи Коши, рассматриваемый в рамках лабораторной работы;
    \item описать соответствующие методы решения задачи:
        \begin{itemize}
            \item метод Пикара;
            \item явный метод Эйлера;
            \item неявный метод Эйлера;
            \item метод Рунге--Кутта 2-го порядка;
        \end{itemize}
	\item описать структуру разрабатываемого ПО;
	\item определить средства программной реализации;
    \item реализовать соответствующее ПО;
	\item привести результаты работы разработанного ПО.
\end{itemize}
