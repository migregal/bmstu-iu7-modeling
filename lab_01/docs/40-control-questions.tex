\chapter{Контрольные вопросы}

\section{Укажите интервалы значений аргумента, в которых можно считать решением заданного уравнения каждое из первых 4-х приближений Пикара. Точность результата оценивать до второй цифры после запятой. Объяснить свой ответ.}

Ответ на данный вопрос даётся на основании значений из таблицы, приведенной выше (Шаг равен $10^{-4}$). 

Учитывая тот факт, что $i$-ое приближение имеет порядок точности $= O(h^{i})$, можно сделать вывод, что метод Пикара $i$-ого порядка точности будет корректен до тех пор, пока его значение совпадает (с определенной степенью точности, в нашем случае - 2 цифры после запятой) с методом Пикара $(i+1)$-ого порядка точности. Как только это значение начинает отклоняться, учитывая тот факт, что более высокое приближение имеет меньшую погрешность, метод нельзя считать корректным решением.

Таким образом:
\begin{enumerate}
    \item 1 приближение считается решением на полуинтервале $[0; 0.85]$;
    \item 2 приближение считается решением на полуинтервале $[0; 1.10]$;
    \item 3 приближение считается решением на полуинтервале $[0; 1.30]$;
    \item 4 приближение считается решением на отрезке $[0; 1.30]$, при этом дать более точную оценку для этого интервала в отсутствие 5-го приближения невозможно;
\end{enumerate}

\section{Пояснить, каким образом можно доказать правильность полученного результата при фиксированном значении аргумента в численных методах.}

Для доказательства корректности полученного результата следует рассматривать малый интервал в окрестности некоторой точки. Если при устремлении шага вычисления к нулю полученные в результате работы алгоритма значения совпадают (с заданной в задаче точностью), то расчет значений был произведен корректно.

\section{Каково значение функции при x=2, т.е. привести значение u(2).}

Примерно \texttt{317.6}